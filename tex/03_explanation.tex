\section{Explication des résultats expérimentaux}
Cette partie répond aux questions 8 à 10. Elle met en correspondance les observations numériques avec les propriétés structurelles de la chaîne.

\subsection{Rappels sur les concepts clé}
\begin{itemize}[label=\textbullet]
  \item États accessibles et classes communicantes.
  \item Classes transitoires/finales et graphe réduit (diagramme de Hasse).
  \item Période d'une classe et apériodicité.
  \item Distributions stationnaires vs. distributions limites.
\end{itemize}
\placeholder{Formuler ces notions avec concision et références au cours.}

\subsection{Question 8 — structure de la chaîne à 27 états}
\begin{itemize}[label=\textbullet]
  \item Identifier les classes communicantes (utiliser le code ASD2, Python ou Matlab).
  \item Tracer le graphe réduit et caractériser chaque classe (transitoire/finale, période).
  \item Fournir les schémas obtenus (inclure les figures générées dans \texttt{images/}).
\end{itemize}
\placeholder{Insérer ici les résultats et figures.}

\subsection{Question 9 — interprétation des questions 1 à 7}
\placeholder{Mettre en regard les trajectoires de \(\Pi^{(n)}\) avec la structure du graphe réduit : quelles classes sont atteintes, quelles périodes expliquent les oscillations, comment les distributions limites se composent.}

\subsection{Question 10 — distributions stationnaires de la classe finale non explorée}
\placeholder{Lister et paramétrer toutes les distributions stationnaires de la classe finale restante (méthode linéaire ou calcul symbolique).}
