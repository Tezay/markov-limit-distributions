\section*{Explication des résultats expérimentaux}
\phantomsection
\addcontentsline{toc}{section}{Explication des résultats expérimentaux}

\subsection*{Question 8 — structure de la chaîne à 27 états}
\phantomsection
\addcontentsline{toc}{subsection}{Question 8 — structure de la chaîne à 27 états}
\begin{itemize}[label=\textbullet]
  \item Identifier les classes communicantes (utiliser le code ASD2, Python ou Matlab).
  \item Tracer le graphe réduit et caractériser chaque classe (transitoire/finale, période).
  \item Fournir les schémas obtenus (inclure les figures générées dans \texttt{images/}).
\end{itemize}
\placeholder{Insérer ici les résultats et figures.}

\subsection*{Question 9 — interprétation des questions 1 à 7}
\phantomsection
\addcontentsline{toc}{subsection}{Question 9 — interprétation des questions 1 à 7}
\placeholder{Mettre en regard les trajectoires de \(\Pi^{(n)}\) avec la structure du graphe réduit : quelles classes sont atteintes, quelles périodes expliquent les oscillations, comment les distributions limites se composent.}

\subsection*{Question 10 — distributions stationnaires de la classe finale non explorée}
\phantomsection
\addcontentsline{toc}{subsection}{Question 10 — distributions stationnaires de la classe finale non explorée}
\placeholder{Lister et paramétrer toutes les distributions stationnaires de la classe finale restante (méthode linéaire ou calcul symbolique).}
