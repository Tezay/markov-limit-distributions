\usepackage[utf8]{inputenc}
\usepackage[T1]{fontenc}
\usepackage{lmodern}
\usepackage[french]{babel}
\usepackage{csquotes}
\usepackage{microtype}
\usepackage{geometry}
\usepackage[backend=biber,style=authoryear,maxbibnames=99]{biblatex}
\usepackage{amsmath,amssymb,amsthm,mathtools}
\usepackage{siunitx}
\usepackage{graphicx}
\usepackage{subcaption}
\usepackage{float}
\usepackage{booktabs}
\usepackage{array}
\usepackage{enumitem}
\usepackage{tikz}
\usepackage{xcolor}
\usepackage{listings}
\usepackage{hyperref}
\usepackage{cleveref}
\usepackage{blkarray}

\usetikzlibrary{automata, positioning, arrows.meta, calc, bending, quotes, shapes.geometric, shadows}

% Tous les noms en "Prénom Nom"
\DeclareNameAlias{default}{given-family}
\DeclareNameAlias{author}{given-family}
\DeclareNameAlias{sortname}{given-family}

\geometry{margin=2.5cm}
\setlength{\parskip}{0.6em}
\setlength{\parindent}{0pt}
\graphicspath{{images/}}

\hypersetup{
  colorlinks=true,
  linkcolor=blue!60!black,
  citecolor=blue!60!black,
  urlcolor=blue!60!black,
  pdftitle={Projet de Probabilités - Chaînes de Markov},
  pdfauthor={Équipe projet}
}

\addbibresource{references.bib}
\setcounter{tocdepth}{2}
\setcounter{secnumdepth}{3}
\sisetup{detect-all}

\lstdefinestyle{codeblock}{
  language=C,
  basicstyle=\ttfamily\small,
  keywordstyle=\color{blue!70!black}\bfseries,
  commentstyle=\color{gray!80},
  stringstyle=\color{green!50!black},
  numberstyle=\tiny\color{gray},
  numbers=left,
  stepnumber=1,
  showstringspaces=false,
  frame=single,
  breaklines=true,
  tabsize=2,
  captionpos=b
}

\theoremstyle{plain}
\newtheorem{theorem}{Théorème}[section]
\newtheorem{lemma}[theorem]{Lemme}
\theoremstyle{definition}
\newtheorem{definition}[theorem]{Définition}
\newtheorem{proposition}[theorem]{Proposition}
\theoremstyle{remark}
\newtheorem{remark}[theorem]{Remarque}

\newcommand{\Prob}{\mathbb{P}}
\newcommand{\EE}{\mathbb{E}}
\newcommand{\Var}{\operatorname{Var}}
\newcommand{\vect}[1]{\boldsymbol{#1}}
\newcommand{\mat}[1]{\mathbf{#1}}
\newcommand{\R}{\mathbb{R}}
\newcommand{\N}{\mathbb{N}}

\newcommand{\placeholder}[1]{\textbf{\color{red}{[À compléter : #1]}}}
