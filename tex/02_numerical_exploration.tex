\section{Exploration numérique}
Cette partie regroupe les réponses aux questions 1 à 7 de l'énoncé, réalisées à l'aide du programme \texttt{Markov Graph Analyzer} développé en C \parencite{markovGraphAnalyzer2025}.

\subsection{Approche de calcul}
\begin{itemize}[label=\textbullet]
  \item Charger la matrice de transition \texttt{input\_data/matrix.txt} et vérifier que chaque ligne somme à 1.
  \item Implémenter la propagation \(\Pi^{(n)} = \Pi^{(0)} P^{n}\) et tracer \(n \mapsto \Pi^{(n)}\) pour repérer les convergences/oscillations.
  \item Conserver les scripts et commandes utilisés (à déposer en annexe ou référencés depuis un dépôt).
\end{itemize}

\subsection{Scénarios et résultats attendus}
Pour chaque configuration initiale ci-dessous, documenter : (i) les vecteurs \(\Pi^{(n)}\) pour \(n=1,2,10,50\), (ii) les graphes de convergence, (iii) l'existence et la valeur éventuelle de la distribution limite.

\subsubsection{Question 1 — départ en état 2}
\placeholder{Tracer \(\Pi_A(n)\), commenter la convergence ou non, et fournir la distribution limite si elle existe.}

\subsubsection{Question 2 — répartition uniforme sur 2, 5, 12, 21, 25}
\placeholder{Analyser la convergence avec les poids initiaux égaux.}

\subsubsection{Question 3 — répartition aléatoire sur 2, 5, 12, 21, 25}
\placeholder{Étudier la sensibilité de la distribution limite aux paramètres \(a,b,c,d,e\).}

\subsubsection{Question 4 — états 8, 9, 16}
\placeholder{Comparer départ en 8 vs. mélange uniforme vs. mélange aléatoire.}

\subsubsection{Question 5 — états 10, 14, 19, 22, 24}
\placeholder{Identifier les comportements limites selon les trois répartitions demandées.}

\subsubsection{Question 6 — états 6, 17, 20}
\placeholder{Documenter les trajectoires de \(\Pi^{(n)}\) et la présence/absence de limite.}

\subsubsection{Question 7 — états 3, 7, 23}
\placeholder{Même analyse : convergence éventuelle et dépendance à l'initialisation.}

\subsection{Visualisations suggérées}
\begin{itemize}[label=\textbullet]
  \item Graphes de barres superposées montrant \(\Pi^{(n)}\) pour quelques valeurs de \(n\).
  \item Courbes de \(\ell_1\)-distance entre \(\Pi^{(n)}\) et la distribution limite candidate.
  \item Schémas de graphe (TikZ) pour les sous-chaînes explorées, si pertinents.
\end{itemize}
