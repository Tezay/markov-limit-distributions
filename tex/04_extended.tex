\section{Pour aller plus loin}
Cette partie reprend les questions 11 à 14 de l'énoncé, centrées sur les preuves et généralisations.

\subsection{Question 11 — combinaisons de distributions limites}
\placeholder{Montrer que \(L_{\Pi_C} = \alpha L_{\Pi_A} + \beta L_{\Pi_B}\), interpréter \(\alpha,\beta\) en probabilités de passage et comparer calcul manuel vs. calcul numérique.}

\subsection{Question 12 — probabilités d'atteindre les classes finales}
\placeholder{Estimer puis calculer précisément \(P(s \to C_k)\) pour \(s \in \{10,19,22,24\}\) et chaque classe finale.}

\subsection{Question 13 — période d'une classe communicante}
\placeholder{Proposer une preuve courte que deux états communicants ont la même période.}

\subsection{Question 14 — distributions limites et stationnaires}
\begin{itemize}[label=\textbullet]
  \item Montrer que l'ensemble des distributions limites \(L\) coïncide avec l'ensemble des distributions stationnaires \(S\).
  \item Discuter la convexité de \(L\).
  \item Prouver l'existence d'au moins une distribution stationnaire pour une chaîne finie irréductible.
  \item Démontrer qu'une chaîne réductible admet une infinité de distributions limites.
  \item Expliquer pourquoi une chaîne irréductible \(d\)-périodique (avec \(d\neq 1\)) peut ne pas admettre de distribution limite.
  \item Citer/comprendre une preuve d'unicité de la distribution limite pour une chaîne finie, irréductible et apériodique.
\end{itemize}
\placeholder{Structurer les preuves de façon concise, en renvoyant aux références appropriées.}
