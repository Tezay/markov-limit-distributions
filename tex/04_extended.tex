\section*{Pour aller plus loin}
\phantomsection
\addcontentsline{toc}{section}{Pour aller plus loin}

\subsection*{Question 11 — Combinaison linéaire des distributions limites depuis l’état 14}
\phantomsection
\addcontentsline{toc}{subsection}{Question 11 — Combinaison linéaire des distributions limites depuis l’état 14}

Comme vu précédemment, l’état \(14\) appartient à la classe \(C_6\), qui est transitoire vers les classes \(C_5\) et \(C_2\).  
Il a une connexion directe avec les états \(2\) et \(8\), donc \(14\) est un état transitoire vers \(C_5\) et \(C_2\).

L’état \(2\) appartient à la classe \(C_2\) et l’état \(8\) appartient à la classe \(C_5\), qui sont toutes deux des classes finales.

Nous cherchons à prouver que la distribution limite \(L\Pi_C\) partant de l’état \(14\) est une combinaison linéaire des distributions limites \(L\Pi_A\) et \(L\Pi_B\) partant respectivement des états \(2\) et \(8\).

\[
\exists (\alpha,\beta) \in \mathbb{R}^2,\quad L\Pi_C = \alpha\,L\Pi_A + \beta\,L\Pi_B.
\]

Nous pouvons donc interpréter les coefficients \(\alpha\) et \(\beta\) comme les probabilités d’atteindre les
classes \(C_5\) et \(C_2\) en partant de l’état \(14\).  
Sachant cela, \(14\) doit nécessairement atteindre la classe \(C_5\) ou \(C_2\), on peut donc en déduire
que :
\[
\alpha + \beta = 1 \quad \text{(somme des probabilités).}
\]

En reprenant l’équation fondamentale qui décrit la dynamique d’une chaîne de Markov.  
Nous définissons :
\[
x_i = P(\text{Atteindre } C_2 \mid X_0 = i),
\qquad
x_j = P(\text{Atteindre } C_2 \mid X_1 = j),
\]
qui découle de l'application de la Formule des Probabilités Totales sur le premier pas de transition.

\bigskip

\textbf{Rappel :} nous pouvons utiliser le premier pas pour par la suite exprimer les limites grâce à  
Homogénéité temporelle : les probabilités de transition pour passer d’un état à un autre en un  
pas sont indépendantes du temps (1.2 4) du sujet).

Nous avons donc l’équation suivante :
\[
x_i = \sum_{j=1}^{27} P(X_1 = j \mid X_0 = i) * x_j
\]

On simplifie :
\[
x_i = \sum_{j=1}^{27} P_{ij} * x_j
\]

\bigskip

En appliquant cela à \(i=14\), et comme \(\alpha = x_{14}\), on obtient :

\[
x_{14} = \alpha
= P_{14,14} \cdot \alpha
+ \sum_{j \in C_2} P_{14,j} \cdot 1
+ \sum_{j \in C_5} P_{14,j} \cdot 0.
\]

C’est-à-dire :
\[
\alpha = P_{14,14}\alpha + \sum_{j\in C_2} P_{14,j}.
\]

\bigskip

\textbf{En isolant \(\alpha\) :}
\[
\alpha - P_{14,14}\alpha = \sum_{j\in C_2} P_{14,j}
\quad\Longleftrightarrow\quad
\alpha (1 - P_{14,14}) = \sum_{j\in C_2} P_{14,j}
\quad\Longleftrightarrow\quad
\alpha = \frac{\displaystyle\sum_{j\in C_2} P_{14,j}}{1 - P_{14,14}}.
\]

\bigskip

On sait que \(P_{14,14} = 0.4\).

De plus :
\[
\sum_{j\in C_2} P_{14,j}
= P_{14,5} + P_{14,12} + P_{14,21} + P_{14,25} + P_{14,2}.
\]

Comme l’état \(14\) ne pointe directement que vers \(2\) et \(5\) :
\[
\sum_{j\in C_2} P_{14,j}
= P_{14,5} + P_{14,2}
= 0.1 + 0.2
= 0.3.
\]

Ainsi :
\[
\alpha = \frac{0.3}{1 - 0.4}
       = \frac{0.3}{0.6}
       = \frac12.
\]

Comme \(\alpha + \beta = 1\), on obtient :
\[
\beta = 1 - \alpha = \frac12.
\]

\[
\boxed{
L\Pi_C = \frac12 L\Pi_A + \frac12 L\Pi_B.
}
\]

\textbf{Conclusion :} la distribution limite issue de l’état \(14\) est égale à la combinaison linéaire à poids égaux des distributions limites issues de \(2\) et \(8\).



\subsection*{Question 12 — probabilités d'atteindre les classes finales}
\phantomsection
\addcontentsline{toc}{subsection}{Question 12 — probabilités d'atteindre les classes finales}
Nous cherchons à déterminer la probabilité d'absorption dans les classes finales \(C_2\) et \(C_5\) en partant des états transitoires \(s \in \{10, 19, 22, 24\}\).
On note \(x_i = P(i \to C_2)\) la probabilité d'atteindre la classe \(C_2\) partant de l'état \(i\).
Puisque les seules classes finales accessibles sont \(C_2\) et \(C_5\), on aura \(P(i \to C_5) = 1 - x_i\).

\paragraph{1. Calcul théorique (Estimation à la main)}

Nous savons d'après la question 11 que pour l'état 14 (classe \(C_6\)), qui mène vers \(C_2\) et \(C_5\) :
\[ x_{14} = P(14 \to C_2) = 0.5 \]

Pour les états de la classe \(C_7\) (\(10, 19, 22, 24\)), nous écrivons le système d'équations linéaires basé sur les transitions en un seul pas, pour arriver jusqu'à \(C_2\) (puis on obtiendra pour \(C_5\) en faisant \(1 - x_i\)) :
\[
\begin{cases}
    x_{10} = 0.4 + 0.2 x_{19} + 0.2 x_{22} + 0.2 x_{24} & (1) \\
    x_{19} = 0.1 x_{10} + 0.5 x_{19} + 0.1 x_{22} + 0.3 x_{24} & (2) \\
    x_{22} = 0.2 x_{10} + 0.2 x_{19} + 0.2 x_{24} & (3) \\
    x_{24} = 0.3 x_{10} + 0.1 x_{14} + 0.3 x_{19} + 0.3 x_{22} & (4)
\end{cases}
\]

\textbf{Résolution :}
En soustrayant l'équation (3) à l'équation (1) on obtient :
\[ x_{10} - x_{22} = 0.4 + 0.2 x_{22} - 0.2 x_{10} \iff 1.2 x_{10} - 1.2 x_{22} = 0.4 \iff x_{10} = x_{22} + \frac{1}{3} \]

En injectant cette relation dans (3) on trouve :
\[ x_{22} = 0.2(x_{22} + \frac{1}{3}) + 0.2 x_{19} + 0.2 x_{24} \]
En utilisant (2) et (4) et en sachant que \(x_{14} = \frac{1}{2}\), la résolution du système donne les valeurs suivantes pour atteindre la classe \(C_2\) :
\[
x_{10} = \frac{2}{3}, \quad
x_{19} = \frac{1}{2}, \quad
x_{22} = \frac{1}{3}, \quad
x_{24} = \frac{1}{2}.
\]

Ensuite, les probabilités d'atteindre \(C_5\) sont les compléments à 1 (car il n'y a que \(C_2\) et \(C_5\) comme classes finales accessibles depuis \(C_7\)) :
\[
P(10 \to C_5) = 1/3, \quad
P(19 \to C_5) = 1/2, \quad
P(22 \to C_5) = 2/3, \quad
P(24 \to C_5) = 1/2
\]

Et donc les autres classes finales \(C_1\) et \(C_3\) sont atteintes avec une probabilité de 0 (car pas reliées à \(C_7\)).

\paragraph{2. Valeur "exacte" par l'ordinateur}

Nous avons exécuté le programme \texttt{markov\_analyzer} pour chacun de ces états de départ avec \(n=100\) étapes. Les commandes utilisées sont :
\begin{verbatim}
./markov_analyzer --in data/moodle/matrix.txt --dist-start 10 --dist-steps 100
./markov_analyzer --in data/moodle/matrix.txt --dist-start 19 --dist-steps 100
./markov_analyzer --in data/moodle/matrix.txt --dist-start 22 --dist-steps 100
./markov_analyzer --in data/moodle/matrix.txt --dist-start 24 --dist-steps 100
\end{verbatim}

Les résultats sont extraits de la ligne \texttt{[Distribution]} des fichiers de sortie (\texttt{data/q12/sim\_*.txt}), qui donne le vecteur \(\Pi^{(100)}\).
Pour obtenir la probabilité d'absorption dans une classe, nous faisons la somme des probabilités des états appartenant à cette classe :
\begin{itemize}
    \item Pour \(C_2 = \{2, 5, 12, 21, 25\}\) : on somme les composantes aux indices 2, 5, 12, 21 et 25.
    \item Pour \(C_5 = \{8, 9, 16\}\) : on somme les composantes aux indices 8, 9 et 16.
\end{itemize}

Les résultats numériques obtenus sont :

\begin{table}[H]
    \centering
    \begin{tabular}{@{}lcccc@{}}
        \toprule
        État de départ \(s\) & 10 & 19 & 22 & 24 \\
        \midrule
        \textbf{Calcul théorique} \(P(s \to C_2)\) & \textbf{0.6667} & \textbf{0.5000} & \textbf{0.3333} & \textbf{0.5000} \\
        Simulation \(\sum_{j \in C_2} \Pi^{(100)}(j)\) & 0.6667 & 0.5000 & 0.3333 & 0.5000 \\
        \midrule
        \textbf{Calcul théorique} \(P(s \to C_5)\) & \textbf{0.3333} & \textbf{0.5000} & \textbf{0.6667} & \textbf{0.5000} \\
        Simulation \(\sum_{j \in C_5} \Pi^{(100)}(j)\) & 0.3333 & 0.5000 & 0.6667 & 0.5000 \\
        \bottomrule
    \end{tabular}
    \caption{Comparaison des probabilités d'absorption théoriques et simulées.}
    \label{tab:q12_absorption}
\end{table}

\textbf{Conclusion :} Les valeurs simulées par l'ordinateur correspondent parfaitement aux valeurs théoriques calculées par résolution du système linéaire. Cela confirme la cohérence de notre modèle et de l'implémentation.

\subsection*{Question 13 : Preuve de l'Égalité des Périodes dans une Classe Communicante}
\phantomsection
\addcontentsline{toc}{subsection}{Question 13 : Preuve de l'Égalité des Périodes dans une Classe Communicante}

\paragraph{Définitions Préalables}

\begin{itemize}[label=$\diamond$]
    \item \textbf{Communication :} Deux états $i$ et $j$ communiquent ($i \longleftrightarrow j$) s'il existe $m \geq 1$ et $k \geq 1$ tels que la probabilité de transition $P_{ij}^{(m)} > 0$ et $P_{ji}^{(k)} > 0$.
    \item \textbf{Période :} La période d'un état $i$, notée $d(i)$, est le plus grand commun diviseur (pgcd) des longueurs de tous les chemins partant de $i$ et y retournant :
    $$d(i) = \text{pgcd} \{n \geq 1 \mid P_{ii}^{(n)} > 0 \}$$
\end{itemize}

\paragraph{Preuve}
Soient $i$ et $j$ deux états de la même classe communicante.
Il existe des chemins :
\begin{itemize}
    \item $i \to j$ de longueur $m$.
    \item $j \to i$ de longueur $k$.
\end{itemize}

Soit $n$ une longueur quelconque d'un chemin de retour pour l'état $j$, i.e., $P_{jj}^{(n)} > 0$. Par définition, $d(j) \mid n$.

Considérons le chemin composite de $i$ à $i$ passant par $j$ :
$$i \xrightarrow{m} j \xrightarrow{n} j \xrightarrow{k} i$$
La longueur totale est $\ell_1 = m + n + k$.
Puisque $P_{ii}^{(m+n+k)} \geq P_{ij}^{(m)} P_{jj}^{(n)} P_{ji}^{(k)} > 0$, $\ell_1$ est une longueur de retour valide pour l'état $i$.

Considérons également le chemin de retour simple :
$$i \xrightarrow{m} j \xrightarrow{k} i$$
La longueur totale est $\ell_2 = m + k$.
Puisque $P_{ii}^{(m+k)} \geq P_{ij}^{(m)} P_{ji}^{(k)} > 0$, $\ell_2$ est une longueur de retour valide pour l'état $i$.

Par définition de $d(i)$, celui-ci doit diviser toutes les longueurs de retour valides, y compris $\ell_1$ et $\ell_2$ :
$$ d(i) \mid (m + n + k) \quad (E1)$$
$$ d(i) \mid (m + k) \quad (E2)$$

En soustrayant $(E1)$ de $(E2)$ (propriété de divisibilité), nous obtenons :
$$d(i) \mid [(m + n + k) - (m + k)]$$
$$d(i) \mid n$$
Cette relation est vraie pour tout $n$ tel que $P_{jj}^{(n)} > 0$. Par conséquent, $d(i)$ divise tous les éléments de l'ensemble dont $d(j)$ est le PGCD. On en déduit :
$$ d(i) \mid d(j) \quad (E3)$$

On peut échanger $$d(j)$$ et $$d(i)$$ et ainsi, on obtient :
$$ d(j) \mid d(i) \quad (E4)$$

Puisque $d(i)$ et $d(j)$ sont des entiers positifs qui se divisent mutuellement, on conclut que :
$$d(i) = d(j)$$
Donc deux états appartenant à la même classe communicante ont nécessairement la même période.



\subsection*{Question 14 — distributions limites et stationnaires}
\phantomsection
\addcontentsline{toc}{subsection}{Question 14 — distributions limites et stationnaires}
\begin{itemize}[label=\textbullet]
  \item Montrer que l'ensemble des distributions limites \(L\) coïncide avec l'ensemble des distributions stationnaires \(S\).
  \item Discuter la convexité de \(L\).
  \item Prouver l'existence d'au moins une distribution stationnaire pour une chaîne finie irréductible.
  \item Démontrer qu'une chaîne réductible admet une infinité de distributions limites.
  \item Expliquer pourquoi une chaîne irréductible \(d\)-périodique (avec \(d\neq 1\)) peut ne pas admettre de distribution limite.
  \item Citer/comprendre une preuve d'unicité de la distribution limite pour une chaîne finie, irréductible et apériodique.
\end{itemize}
\placeholder{Structurer les preuves de façon concise, en renvoyant aux références appropriées.}
