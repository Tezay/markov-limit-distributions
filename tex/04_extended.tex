\section*{Pour aller plus loin}
\phantomsection
\addcontentsline{toc}{section}{Pour aller plus loin}

\subsection*{Question 11 — combinaisons de distributions limites}
\phantomsection
\addcontentsline{toc}{subsection}{Question 11 — combinaisons de distributions limites}
\placeholder{Montrer que \(L_{\Pi_C} = \alpha L_{\Pi_A} + \beta L_{\Pi_B}\), interpréter \(\alpha,\beta\) en probabilités de passage et comparer calcul manuel vs. calcul numérique.}

\subsection*{Question 12 — probabilités d'atteindre les classes finales}
\phantomsection
\addcontentsline{toc}{subsection}{Question 12 — probabilités d'atteindre les classes finales}
\placeholder{Estimer puis calculer précisément \(P(s \to C_k)\) pour \(s \in \{10,19,22,24\}\) et chaque classe finale.}

\subsection*{Question 13 : Preuve de l'Égalité des Périodes dans une Classe Communicante}
\phantomsection
\addcontentsline{toc}{subsection}{Question 13 : Preuve de l'Égalité des Périodes dans une Classe Communicante}

\paragraph{Définitions Préalables}

\begin{itemize}[label=$\diamond$]
    \item \textbf{Communication :} Deux états $i$ et $j$ communiquent ($i \longleftrightarrow j$) s'il existe $m \geq 1$ et $k \geq 1$ tels que la probabilité de transition $P_{ij}^{(m)} > 0$ et $P_{ji}^{(k)} > 0$.
    \item \textbf{Période :} La période d'un état $i$, notée $d(i)$, est le plus grand commun diviseur (pgcd) des longueurs de tous les chemins partant de $i$ et y retournant :
    $$d(i) = \text{pgcd} \{n \geq 1 \mid P_{ii}^{(n)} > 0 \}$$
\end{itemize}

\paragraph{Preuve}
Soient $i$ et $j$ deux états de la même classe communicante.
Il existe des chemins :
\begin{itemize}
    \item $i \to j$ de longueur $m$.
    \item $j \to i$ de longueur $k$.
\end{itemize}

Soit $n$ une longueur quelconque d'un chemin de retour pour l'état $j$, i.e., $P_{jj}^{(n)} > 0$. Par définition, $d(j) \mid n$.

Considérons le chemin composite de $i$ à $i$ passant par $j$ :
$$i \xrightarrow{m} j \xrightarrow{n} j \xrightarrow{k} i$$
La longueur totale est $\ell_1 = m + n + k$.
Puisque $P_{ii}^{(m+n+k)} \geq P_{ij}^{(m)} P_{jj}^{(n)} P_{ji}^{(k)} > 0$, $\ell_1$ est une longueur de retour valide pour l'état $i$.

Considérons également le chemin de retour simple :
$$i \xrightarrow{m} j \xrightarrow{k} i$$
La longueur totale est $\ell_2 = m + k$.
Puisque $P_{ii}^{(m+k)} \geq P_{ij}^{(m)} P_{ji}^{(k)} > 0$, $\ell_2$ est une longueur de retour valide pour l'état $i$.

Par définition de $d(i)$, celui-ci doit diviser toutes les longueurs de retour valides, y compris $\ell_1$ et $\ell_2$ :
$$ d(i) \mid (m + n + k) \quad (E1)$$
$$ d(i) \mid (m + k) \quad (E2)$$

En soustrayant $(E1)$ de $(E2)$ (propriété de divisibilité), nous obtenons :
$$d(i) \mid [(m + n + k) - (m + k)]$$
$$d(i) \mid n$$
Cette relation est vraie pour tout $n$ tel que $P_{jj}^{(n)} > 0$. Par conséquent, $d(i)$ divise tous les éléments de l'ensemble dont $d(j)$ est le PGCD. On en déduit :
$$ d(i) \mid d(j) \quad (E3)$$

On peut échanger $$d(j)$$ et $$d(i)$$ et ainsi, on obtient :
$$ d(j) \mid d(i) \quad (E4)$$

Puisque $d(i)$ et $d(j)$ sont des entiers positifs qui se divisent mutuellement, on conclut que :
$$d(i) = d(j)$$
Donc deux états appartenant à la même classe communicante ont nécessairement la même période.



\subsection*{Question 14 — distributions limites et stationnaires}
\phantomsection
\addcontentsline{toc}{subsection}{Question 14 — distributions limites et stationnaires}
\begin{itemize}[label=\textbullet]
  \item Montrer que l'ensemble des distributions limites \(L\) coïncide avec l'ensemble des distributions stationnaires \(S\).
  \item Discuter la convexité de \(L\).
  \item Prouver l'existence d'au moins une distribution stationnaire pour une chaîne finie irréductible.
  \item Démontrer qu'une chaîne réductible admet une infinité de distributions limites.
  \item Expliquer pourquoi une chaîne irréductible \(d\)-périodique (avec \(d\neq 1\)) peut ne pas admettre de distribution limite.
  \item Citer/comprendre une preuve d'unicité de la distribution limite pour une chaîne finie, irréductible et apériodique.
\end{itemize}
\placeholder{Structurer les preuves de façon concise, en renvoyant aux références appropriées.}
