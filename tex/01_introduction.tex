\section{Introduction}

\subsection{Contexte}
Ce projet s'inscrit dans le cours de probabilités (SM301) et vise à étudier le comportement à long terme de chaînes de Markov finies.
Nous analyserons une chaîne particulière de 27 états et la mobilisation des concepts de classes communicantes, distributions stationnaires et distributions limites.

Nous avons déjà abordé les fondements théoriques des chaînes de Markov lors d'un précédent projet dans le module d'Algorithmique et Structure de données 2 au travers de simulations numériques, et ce projet permettra de mettre en pratique ces notions et résultats.

Les chaînes de Markov sont des modèles probabilistes largement utilisés en sciences, ingénierie et économie pour modéliser des systèmes évoluant de manière aléatoire dans le temps.
Comprendre leur comportement à long terme est crucial pour diverses applications, telles que la modélisation des files d'attente, les algorithmes de recommandation, PageRank de Google, routage dans les réseaux etc.
\parencite{veritasiumMarkov2025}

\subsection{Objectifs du projet}
\begin{itemize}[label=\textbullet]
  \item Explorer expérimentalement une chaîne de Markov à 27 états (questions 1 à 7).
  \item Interpréter les résultats à l'aide des notions théoriques (questions 8 à 10).
  \item Aller plus loin avec des preuves et des analyses complémentaires (questions 11 à 14).
\end{itemize}

\subsection{Méthodologie et Ressources}
Pour faciliter le suivi de ce projet et la transparence du rapport, le binaire du programme C \texttt{markov\_analyzer}, les fichiers de sortie bruts, ainsi que l'historique des versions de ce document PDF sont consultables sur notre dépôt GitHub public :
\begin{center}
    \url{https://github.com/Tezay/markov-limit-distributions}
\end{center}
Le code source du programme C \texttt{markov\_analyzer} est également disponible sur ce dépôt GitHub \parencite{markovGraphAnalyzer2025} (les fonctions principales utilisées restent disponibles en annexe) :
\begin{center}
    \url{https://github.com/Tezay/markov-graph-analyzer}
\end{center}

\newpage
