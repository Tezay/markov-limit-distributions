\section{Introduction}

\subsection{Contexte}
Ce projet s'inscrit dans le cours de probabilités (SM301) et vise à étudier le comportement à long terme de chaînes de Markov finies. Nous analyserons une chaîne particulière de 27 états et la mobilisation des concepts de classes communicantes, distributions stationnaires et distributions limites.

% A compléter
\placeholder{Présenter brièvement l'intérêt applicatif des chaînes de Markov et les liens avec le cours d'Algorithmique et Structure de données 2.}

\subsection{Objectifs du projet}
\begin{itemize}[label=\textbullet]
  \item Explorer expérimentalement une chaîne de Markov à 27 états (questions 1 à 7).
  \item Interpréter les résultats à l'aide des notions théoriques (questions 8 à 10).
  \item Aller plus loin avec des preuves et des analyses complémentaires (questions 11 à 14).
  \item Produire un rapport concis (20 pages max) et une présentation orale de 10--15 minutes.
\end{itemize}

\subsection{Organisation du rapport}
Chaque partie du document correspond aux sections de l'énoncé fourni. Les questions sont regroupées pour faciliter la rédaction progressive et la validation des résultats.
\begin{itemize}[label=\textbullet]
  \item \textbf{Exploration numérique} : implémentations et sorties expérimentales sur la chaîne de 27 états.
  \item \textbf{Explication théorique} : concepts clés (accessibilité, classes, période, distributions stationnaires/limites).
  \item \textbf{Pour aller plus loin} : questions de preuve et généralisations.
  \item \textbf{Annexes} : extraits de code, matrices utilisées, graphiques supplémentaires.
\end{itemize}

\subsection{Jeu de données et ressources}
\begin{itemize}[label=\textbullet]
  \item \textbf{Matrice de transition} : fichier texte \texttt{input\_data/matrix.txt} fourni (même format que le projet ASD2).
  \item \textbf{Schéma d'exemple} : \texttt{images/matrix\_proba.pdf} disponible pour illustrer une structure de transitions.
  \item \textbf{Programmes} : réutiliser le module \texttt{matrix.c/matrix.h} ou équivalent (ASD2) pour les calculs \parencite{markovGraphAnalyzer2025}.
\end{itemize}

\subsection{Méthodologie proposée}
\begin{itemize}[label=\textbullet]
  \item Automatiser les simulations (Python/Matlab/C) pour générer les vecteurs de probabilité et les graphes de convergence.
  \item Documenter chaque configuration initiale et paramètre exploré.
  \item Conserver les figures et résultats bruts pour relecture en annexe, et ne garder dans le corps du texte que les visualisations synthétiques.
\end{itemize}
